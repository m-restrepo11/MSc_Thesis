% the abstract
\textbf{English}\\

Quantum electrodynamics in two dimensions (one spatial dimension + time) with Dirac fermions was a model first studied by Julian Schwinger in 1962 and thus it was coined the Schwinger model. The massless model is completely solvable and shows very interesting physical phenomena such as a chiral anomaly, a massive boson generated via a kind of dynamical Higgs mechanism,  the presence of confinement analogous to quark confinement in QCD$_4$ and spontaneous symmetry breaking of a $U(1)$ axial symmetry. The Schwinger model is the simplest gauge theory being completely solvable, thus, it is a rich toy model for studying more complex gauge theories.

In this work, we quantize the Schwinger model using Hamiltonian quantization in the temporal Weyl gauge ($A0=0$) which then can be written as a lattice system of spins using Kogut-Susskind fermions and a Jordan-Wigner transformation. This lattice model is numerically studied using exact diagonalization methods. We study the ground state energy,  the first few excited states energies, the existence of a gap and the order parameter values both in the free massless and massive model. Additionally, we explore the dynamical evolution of the spectrum of the Dirac operator, the dynamics of the ground state and the behavior of order parameters such as the chiral condensate density as a consequence of the introduction of global quenches in the system in both the massless and the massive interacting models in the presence of a background electric field.\vspace{30 pt}


\textbf{Español}\\

La electrodinámica cuántica en dos dimensiones (una dimensión espacial + tiempo) con fermiones de Dirac es un modelo que estudió por primera vez Julian Schwinger en 1962, y por lo tanto se conoce como el modelo de Schwinger. El modelo sin masa es completamente soluble y exhibe fenómenos físicos muy interesantes como lo son una anomalía quiral, un bosón masivo generado por un tipo de mecanismo de Higgs dinámico, presencia de confinamiento análogo al confinamiento de los quarks en QCD$_4$ y el rompimiento espontáneo de una simetría $U(1)$ axial. El modelo de Schwinger es la teoría gauge más simple, y por lo tanto, es un modelo de juguete muy útil para el estudio de teorías gauge más complejas.


En este trabajo, cuantizamos el modelo usando cuantización Hamiltoniana en el gauge temporal de Weyl ($A0=0$), y posteriormente escribimos el modelo como una cadena de espines usando fermiones de Kogut-Susskind y una transformación de Jordan-Wigner. Este modelo discreto se estudia numéricamente usando métodos de diagonalización exacta. Estudiamos la energía del estado base, las energías de unos pocos estados excitados, la existencia de un gap en el espectro y parámetros de orden tanto en el modelo libre sin masa como en el modelo masivo. Adicionalmente, exploramos la dinámica del espectro del operador de Dirac, la dinámica del estado base y el comportamiento de parámetros de orden como la densidad del condensado quiral como consecuencia de la introducción de un quench global tanto en el modelo no-masivo como en el modelo masivo e interactuante en presencia de un campo eléctrico de fondo.