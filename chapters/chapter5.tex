\chapter{Discussion and outlook}

In this thesis, we studied the Schwinger model both in its continuum and discrete lattice versions. We obtained various important and interesting results both theoretically and numerically regarding the chiral anomaly, the spectrum and the dynamics of the model.\\

We expressed the lattice version of the Schwinger model as a spin chain model with long-range spin-spin interactions modeling Coulomb's law. We studied the free massless, free massive and fully interacting models numerically using the exact diagonalization method. This was programmed in python using the package Quspin. We computed the ground state energy density, the gap energy, the chiral condensate density order parameter, and the axial fermion density order parameter. In particular, these order parameters were studied in the presence of a background electric field which induces a critical behavior in the system as a consequence of the dynamics of pair production in one spatial dimension. Finally, some simple global quantum quenches were simulated and the Loschmidt echo density and the chiral density parameter and axial fermion density parameter were studied in this non-equilibrium context.\\

Perhaps the most natural continuation of this work should be towards studying more realistic gauge theories such as QCD using the same methods outlined in this work. It is important to note that we benefited greatly from working in 1+1 dimensional space, and higher dimensional theories carry more difficulties an subtleties. Another possibility instead of increasing dimensionality could be to consider another gauge theory with a different gauge group. The Schwinger model being an abelian $U(1)$ theory is the simplest gauge theory and perhaps the introduction of a non-abelian theory can shed light into novel and interesting results. Finally, we could think of continuing the venture along out of equilibrium dynamics of the Schwinger model. In particular, exploration of so-called dynamical phase transitions induced by quench dynamics in the Schwinger model. There is still a lot of ground to cover and the model despite its simplicity is full of interesting physics and neat, unexpected results.